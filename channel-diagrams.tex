% https://tex.stackexchange.com/a/55604

\usepackage{tikz}
\usetikzlibrary{shapes,backgrounds,calc}
\usetikzlibrary{positioning, shapes.geometric}

\makeatletter
\tikzset{fill halves/.style  args={#1,#2}{%
  circle,
  postaction={%
    insert path={
      \pgfextra{% 
        % This entire script assumes that we're working with a circle, making use of the anchors
        % we expect to find on a circle
        % Calculates "insiderad" by looking at the distance from the center to the east anchor
        \pgfpointdiff{\pgfpointanchor{\pgf@node@name}{center}}%
                    {\pgfpointanchor{\pgf@node@name}{east}}%            
        \pgfmathsetmacro\insiderad{\pgf@x}
        % We start at the east anchor of the node and then move in by "pgflinewidth"
        % then we draw an arc from 0 to 180
        \fill[#1] (\pgf@node@name.base) ([xshift=-\pgflinewidth/sqrt(2), yshift=-\pgflinewidth/sqrt(2)]\pgf@node@name.north east) arc
                          (45:225:\insiderad-\pgflinewidth)--cycle;
        \fill[#2] (\pgf@node@name.base) ([xshift=\pgflinewidth/sqrt(2), yshift=\pgflinewidth/sqrt(2)]\pgf@node@name.south west)  arc
                            (225:405:\insiderad-\pgflinewidth)--cycle;
        \draw[-] (\pgf@node@name.north east) to (\pgf@node@name.south west);
      }
    }
  }
}
}  
\makeatother  

\makeatletter
\tikzset{fill thirds/.style  args={#1,#2,#3}{%
  circle,
  postaction={%
    insert path={
      \pgfextra{% 
        % This entire script assumes that we're working with a circle, making use of the anchors
        % we expect to find on a circle
        % Calculates "insiderad" by looking at the distance from the center to the east anchor
        \pgfpointdiff{\pgfpointanchor{\pgf@node@name}{center}}%
                    {\pgfpointanchor{\pgf@node@name}{east}}%            
        \pgfmathsetmacro\insiderad{\pgf@x - \pgflinewidth}
        % We start at the east anchor of the node and then move in by "pgflinewidth"
        % then we draw an arc from 0 to 180
        \fill[#1] (\pgf@node@name.center) -- ++ (90:\insiderad pt) arc (90:210:\insiderad pt) --cycle;
        \fill[#2] (\pgf@node@name.center) -- ++ (210:\insiderad pt) arc (210:330:\insiderad pt) --cycle;
        \fill[#3] (\pgf@node@name.center) -- ++ (330:\insiderad pt) arc (330:450:\insiderad pt) --cycle;
        \draw (\pgf@node@name.center) -- ++ (90:\insiderad pt);
        \draw (\pgf@node@name.center) -- ++ (210:\insiderad pt);
        \draw (\pgf@node@name.center) -- ++ (330:\insiderad pt);
      }
    }
  }
}
}  
\makeatother  

\makeatletter
\tikzset{guarantee thirds/.style  args={#1,#2,#3}{%
  semicircle,
  rotate=180,
  postaction={%
    insert path={
      \pgfextra{% 
        % This entire script assumes that we're working with a circle, making use of the anchors
        % we expect to find on a circle
        % Calculates "insiderad" by looking at the distance from the center to the east anchor
        \pgfpointdiff{\pgfpointanchor{\pgf@node@name}{south}}%
                    {\pgfpointanchor{\pgf@node@name}{north}}%            
        \pgfmathsetmacro\insiderad{\pgf@y-1.5\pgflinewidth}
        % We start at the east anchor of the node and then move in by "pgflinewidth"
        % then we draw an arc from 0 to 180
        \fill[#1] ([yshift=\pgflinewidth/2]\pgf@node@name.south) -- ++ (0:\insiderad pt) arc (0:60:\insiderad pt) --cycle;
        \fill[#2] ([yshift=\pgflinewidth/2]\pgf@node@name.south) -- ++ (60:\insiderad pt) arc (60:120:\insiderad pt) --cycle;
        \fill[#3] ([yshift=\pgflinewidth/2]\pgf@node@name.south) -- ++ (120:\insiderad pt) arc (120:180:\insiderad pt) --cycle;
        \draw ([yshift=\pgflinewidth/2]\pgf@node@name.south) -- ++ (0:\insiderad pt);
        \draw ([yshift=\pgflinewidth/2]\pgf@node@name.south) -- ++ (60:\insiderad pt);
        \draw ([yshift=\pgflinewidth/2]\pgf@node@name.south) -- ++ (120:\insiderad pt);
        \draw ([yshift=\pgflinewidth/2]\pgf@node@name.south) -- ++ (180:\insiderad pt);
      }
    }
  }
}
}  
\makeatother  



\tikzset{
  circA/.style={ draw, circle, fill=red!50, minimum size=4mm }
}

\tikzset{
  circB/.style={ draw, circle, fill=blue!50, minimum size=4mm  }
}

\tikzset{
  circAB/.style={ draw, circle, fill halves={blue!50,red!50}, minimum size=4mm }
}
\tikzset{
  circAI/.style={ draw, circle, fill halves={blue!50,green!50}, minimum size=4mm }
}
\tikzset{
  circBI/.style={ draw, circle, fill halves={red!50,green!50}, minimum size=4mm }
}
\tikzset{
  circABI/.style={ draw, semicircle, rotate=180, minimum size=2mm }
}
\tikzset{
  guarAI/.style={ draw, semicircle, minimum size=2mm }
}
\tikzset{
  guarBI/.style={ draw, semicircle, minimum size=2mm }
}

\tikzset{
  sqadj/.style={ draw, minimum size=4mm }
}
