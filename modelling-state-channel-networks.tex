
\section{Modelling State Channel Networks}

In this section we will present a simple model for state channel networks.
The model is intended to be easy to understand and reason about, while still capturing the essential features of state channel networks.
It will form the basis for the protocols introduced later in the paper, along with the tools for proving the correctness of their constructions.

\subsection{A System of Balances}

At the heart of our model lies a simple system of balances. 
We start by describing that system.

In this paper, we simplify the explanation by only considering a single asset, which we will refer to as \textbf{coins}.
We will further simplify matters by specifying that quantities of coins will have no maximum
size, taking values in $\mathbb{Z}^+$.
This allows us to avoid dealing with integer overflows when presenting operations.
These simplifications do not cause any limitations in practice and all the work
here can be applied to state channel networks that manage an arbitrary number of 
real-world assets.

In order to store value, a state channel network must be backed by assets held on-chain.
In our explanation, we assume that these funds are held and managed by a single\footnote{
In practice, the adjudicator functionality could be split across multiple smart contracts.
} smart contract,
which we will refer to as the \textbf{adjudicator}.

The first purpose of the adjudicator is to store the balance of coins held for a given address.
Addresses can correspond either to participants in the network or to state channels.
A \textbf{participant address} is a regular blockchain address, generated in the standard way from the signature scheme.
A \textbf{channel address} is formed by taking the hash of the participant addresses along with a nonce, $k$, that is chosen by the participants in order to distinguish their channels from one another.
Given an arbitrary address, $A$, we assume that the properties of the signature scheme and hashing algorithm make it neither possible to find a private key for $A$ nor to find a channel whose address is $A$.

\begin{figure}[h]\centering
  \makebox[\textwidth][c]{\begin{tikzpicture}[x=3.5cm,y=1.5cm]
  \node[draw=black, rounded corners=0.5cm, outer sep=0.2cm] (adj0) {
    \begin{tabular}{c|c}
      \multicolumn{2}{c}{\textbf{Adjudicator}} \\
      Address & Balance \\
      \hline
      Bob & 5 \\
      Alice & 2 \\
    \end{tabular}
  };

  \node[right=of adj0] (not0) { $\adj{\holds{\text{Bob}}{5}{}, \;\holds{\text{Alice}}{2}{}}$ };

  \node[right=of not0] (diag0) {
    \begin{tikzpicture}[y=1cm, x=1.5cm]
      \node[sqadj] (a0) at (0, 0) {};
      \node[circB, label=below:Bob] (p0) at (0, -1) {};
      \node[sqadj] (a1) at (1, 0) {};
      \node[circA, label=below:Alice] (p1) at (1, -1) {};
      \draw[->] (a0) edge node[midway, left] { $5$ } (p0);
      \draw[->] (a1) edge node[midway, left] { $2$ } (p1);
    \end{tikzpicture}
  };

  \draw ([yshift=-1.3cm]adj0.west) to[out=270,in=90] ([yshift=-2.1cm]adj0.center)node[yshift=-0.3cm] {on-chain view} to[in=270, out=90] ([yshift=-1.3cm]adj0.east);
  \draw ([yshift=-1.3cm]not0.west) to[out=270,in=90] ([yshift=-2.1cm]not0.center)node[yshift=-0.3cm] {channel notation} to[in=270, out=90] ([yshift=-1.3cm]not0.east);
  \draw ([yshift=-1.3cm]diag0.west) to[out=270,in=90] ([yshift=-2.1cm]diag0.center)node[yshift=-0.3cm] {outcome diagram} to[in=270, out=90] ([yshift=-1.3cm]diag0.east);


\end{tikzpicture}
}
  \caption{
    Three different ways of representing the situation where Bob has $5$ coins
    stored against his address in the adjudicator and Alice has $2$.
  }\label{fig:balance-notation}
\end{figure}

We model the adjudicator as having a simple mapping that stores the quantity of coins for
an address.
If an address does not appear in the table we take the balance to be zero.
Figure \ref{fig:balance-notation} introduces the notation we will use to describe the system.

The \textbf{deposit} operation, $D_A(x)$, is an on-chain operation used to increase $A$'s balance by $x$ coins.
There are no restrictions on who can deposit coins for an address, but the
transaction must always include a transfer of $x$ coins into the adjudicator.

The \textbf{withdrawal} operation, $W_A(x)$, can be used to withdraw coins held at participant address, $A$, by any
party with the knowledge of the corresponding private key. 
In practice the withdrawal should also specify the blockchain address where the funds should be sent.
A potential method signature is \texttt{withdraw(fromAddr, toAddr, amount, signature)}, 
where \texttt{signature} is $A$'s signature of the other parameters
\footnote{In practice, we could add the \texttt{senderAddress} to the parameters to sign,
in order to prevent replay attacks by other parties.}.

\begin{figure}[h]\centering
  \makebox[\textwidth][c]{  \begin{tikzpicture}[x=5.5cm,y=1.5cm]
    \node[draw=black, rounded corners=0.5cm, outer sep=0.2cm] (adj0) at (0, 0) {
      \begin{tabular}{c|c}
        \multicolumn{2}{c}{\textbf{Adjudicator}} \\
        Address & Balance \\
        \hline
        Bob & 5 \\
        & \\
      \end{tabular}
    };

    \node[draw=black, rounded corners=0.5cm, outer sep=0.2cm] (adj1) at (1, 0) {
      \begin{tabular}{c|c}
        \multicolumn{2}{c}{\textbf{Adjudicator}} \\
        Address & Balance \\
        \hline
        Bob & 5 \\
        Alice & 4 \\
      \end{tabular}
    };

    \node[draw=black, rounded corners=0.5cm, outer sep=0.2cm] (adj2) at (2, 0) {
      \begin{tabular}{c|c}
        \multicolumn{2}{c}{\textbf{Adjudicator}} \\
        Address & Balance \\
        \hline
        Bob & 2 \\
        Alice & 4 \\
      \end{tabular}
    };

    \node (not0) at (0, -1) {$\adj{\holds{\text{Bob}}{5}{}}$ };
    \node (not0) at (1, -1) {$\adj{\holds{\text{Bob}}{5}{},\;\holds{\text{Alice}}{4}{}}$ };
    \node (not0) at (2, -1) {$\adj{\holds{\text{Bob}}{2}{},\;\holds{\text{Alice}}{4}{}}$ };

    \node (diag0) at (0, -2) {
      \begin{tikzpicture}[y=1cm]
        \node[sqadj] (a0) at (0, 0) {};
        \node[circB, label=below:Bob] (p0) at (0, -1) {};
        \draw[->] (a0) edge node[midway, left] { $5$ } (p0);
      \end{tikzpicture}
    };

    \node (diag1) at (1, -2) {
      \begin{tikzpicture}[y=1cm, x=1cm]
        \node[sqadj] (a0) at (0, 0) {};
        \node[circB, label=below:Bob] (p0) at (0, -1) {};
        \node[sqadj] (a1) at (1, 0) {};
        \node[circA, label=below:Alice] (p1) at (1, -1) {};
        \draw[->] (a0) edge node[midway, left] { $5$ } (p0);
        \draw[->] (a1) edge node[midway, left] { $4$ } (p1);
      \end{tikzpicture}
    };

    \node (diag2) at (2, -2) {
      \begin{tikzpicture}[y=1cm, x=1cm]
        \node[sqadj] (a0) at (0, 0) {};
        \node[circB, label=below:Bob] (p0) at (0, -1) {};
        \node[sqadj] (a1) at (1, 0) {};
        \node[circA, label=below:Alice] (p1) at (1, -1) {};
        \draw[->] (a0) edge node[midway, left] { $2$ } (p0);
        \draw[->] (a1) edge node[midway, left] { $4$ } (p1);
      \end{tikzpicture}
    };

    \draw[->] (adj0) edge node[midway, above] { $D_\text{Alice}(4)$} (adj1);
    \draw[->] (adj1) edge node[midway, above] { $W_\text{Bob}(3)$} (adj2);
  \end{tikzpicture}
}
  \caption{Deposits and withdrawals.}\label{fig:deposit-withdrawal}
\end{figure}

To recap, we now have a simple smart contract that can store a balance against an address, which either represents a participant or a channel.
The balances can be increased and decreased through deposits and withdrawals.
Anyone can deposit into an address of either type\footnote{But there is nothing to be gained from depositing into a participant address.} but withdrawals can only occur from 
participant addresses - and only by a party who knows the private key.
The total coins held by the smart contract should always equal the sum of the balances.

\subsection{State Channel Outcomes}

In our model, we think of a state channel as a off-chain protocol followed by a set of participants,
enabling them to reach an \textbf{outcome} that can be used to change the balances on the chain.
In general, the outcome can be an arbitrary data structure, the format and interpretation of which
is specified by the state channel network protocol being used.
For example, the \textit{allocation} is a simple type of outcome used by both Turbo and Nitro
protocols.
The data in an allocation represents a list of addresses paired with totals,
specifying how the channel's balance should be distributed.

\begin{figure}[h]\centering
  \makebox[\textwidth][c]{\begin{tikzpicture}[x=1cm,y=1cm]
  \node[draw=black, rounded corners=0.5cm, outer sep=0.2cm] (adj0) at (0, 0) {
    \begin{tabular}{c|c|c}
      \multicolumn{3}{c}{\textbf{Adjudicator}} \\
      Address & Balance & Outcome \\
      \hline
      $L$ & $5$ & $A$: 1, $B$: 2 \\
      $L'$ & $6$ & \\
      $L''$ &  & $A$: 3, $B$: 4 \\
    \end{tabular}
  };

  \node (not0) at (3, -2.4) { $\adj{\holds{\text{L}}{5}{\alloc{A:1, B:2}}, \;\holds{L'}{6}{}, \; \holds{L''}{}{\alloc{A:3,B:4}}}$ };

  \node (diag0) at (7, 0) {
    \begin{tikzpicture}[y=1cm, x=1.5cm]
      \node[sqadj] (adj0) at (0, 0) {};
      \node[circAB, label=left:$L$] (L) at (0, -1) {};
      \draw[->] (adj0) edge node[midway, left] { $5$ } (L);
      \node[circA, label=below:$A$] (LA) at (-0.5, -2) {};
      \node[circB, label=below:$B$] (LB) at (0.5, -2) {};
      \draw[->] (L) edge node[midway, left] { $1$ } (LA);
      \draw[->] (L) edge node[midway, right] { $2$ } (LB);

      \node[sqadj] (a1) at (1, 0) {};
      \node[circAB, label=below:$L'$] (p1) at (1, -1) {};
      \draw[->] (a1) edge node[midway, left] { $6$ } (p1);

      \node[circAB, label=right:$L''$] (L'') at (2, -1) {};
      \node[circA, label=below:$A$] (L''A) at (1.5, -2) {};
      \node[circB, label=below:$B$] (L''B) at (2.5, -2) {};
      \draw[->] (L'') edge node[midway, left] { $3$ } (L''A);
      \draw[->] (L'') edge node[midway, right] { $4$ } (L''B);
    \end{tikzpicture}
  };

\end{tikzpicture}
}
  \caption{
    Representation of (allocation) outcomes in the three different diagram formats.
    We show the three possible cases: a channel, $L$, with both a balance and an outcome;
    a channel, $L'$, with a balance but no outcome; 
    and a channel, $L''$, with an outcome but no balance.
    In colouring the channels, we have assumed the participants in each channel are $A$ and $B$.
  }\label{fig:outcome-notation}
\end{figure} 

We separate the process of updating the balances according to the channel outcome into two stages:
finalization and redistribution.
\textbf{Finalization} is the process of getting the outcome of the state channel into the
adjudicator on chain, where it will be stored against the channel address.
\textbf{Redistribution} is the process of updating the balances in the adjudicator, according
to the outcome.

How an outcome is finalized is a question of the rules of operation of a state channel.
The rules of operation are somewhat orthogonal to the network protocols that are the focus of this paper.
The ForceMove protocol is an example of a set of state channel operation rules.
The protocols presented here are known to work with ForceMove channels but could also likely
be made to work with other operation rules.

\begin{figure}[h]\centering
  \begin{tikzpicture}[x=4cm,y=2cm]
  \tikzstyle{every node}=[draw, ellipse]

  % Specification of nodes (position, etc.)
  \node (Setup) at (-0.8,0) {Setup};
  \node (Running) at (0,0) {Running};
  \node (Concluded) at (1, 0) {Concluded};
  \node (Challenge) at (0.5, 1) {Challenge};
  \node (Finalized) at (1.5, 1) {Finalized};

  \begin{scope}[-{Latex[length=2mm]}]
    \tikzstyle{every node}=[draw=none,below]
    \draw (Setup) to (Running);
    \draw (Running) to (Concluded);
    \draw (Concluded) edge[bend right] node[midway, right] {register} (Finalized);
    \draw (Running) edge[bend left] node[midway,above left] {force-move} (Challenge) ;
    \draw (Challenge) edge[bend left] node[right] {respond} (Running);
    \draw (Challenge) edge node[midway, below] {timeout} (Finalized) ;
  \end{scope}
\end{tikzpicture}


  \caption{
    ForceMove Channel Operation.
    The parts of the adjudicator responsible for storing challenges are omitted from the diagram.
  }\label{fig:modes}
\end{figure}

In ForceMove, there are two ways for an outcome to be finalized.
The first corresponds to a non-collaborative closing of the channel:
one participant launches an on-chain challenge by providing a sequence of signed statements to the force-move operation;
this starts a timeout period;
if no other participant responds during timeout period, then the challenge times out and the outcome
corresponding to the challenge state is finalized.
The second corresponds to a collaborative closing of the channel:
all the participants sign a special conclude state, which contains the channel outcome to
produce an object called a conclusion proof; 
any participant can then register this outcome on-chain to create a finalized outcome.
By closing the channel collaboratively the participants avoid having to wait for the challenge period.

For the purpose of the model, a crucial property is that the rules of operation only allow one outcome to be finalized for each channel.
As we will see in section \ref{sec:reasoning}, it is also important that the rules of operation make it possible to know which outcome(s) a participant can finalize from a given state.

\subsection{Redistribution}

The second part of extracting the funds from a state channel is the redistribution step.
Redistribution involves calling a sequence of on-chain \textbf{operations} to manipulate
the balances and finalized outcomes.
The allowed operations are defined by the network protocol used.
Figure \ref{fig:redistribution} shows an example of the transfer operation from Turbo protocol.

\begin{figure}[h]\centering
  \makebox[\textwidth][c]{
\begin{tikzpicture}[x=8cm, y=2cm]
  \node[draw=black, rounded corners=0.5cm] (adj0) at (0, 0) {
    \begin{tabular}{c|c|c}
      \multicolumn{3}{c}{\textbf{Adjudicator}} \\
      Address & Balance & Outcome \\
      \hline
      L & 5 & Bob: 2, Alice: 3 \\
      Bob & 4 \\

    \end{tabular}
  };

  \node[draw=black, rounded corners=0.5cm] (adj0) at (1, 0) {
    \begin{tabular}{c|c|c}
      \multicolumn{3}{c}{\textbf{Adjudicator}} \\
      Address & Balance & Outcome \\
      \hline
      L & 5 & Bob: 2 \\
      Bob & 4 \\
      Alice & 3 \\
    \end{tabular}
  };

  \node (not0) at (0, -1) { $\adj{\holds{L}{5}{\alloc{\text{Bob}: 2, \text{Alice}: 3}}, \;\holds{\text{Bob}}{4}{}}$ };
  \node (not0) at (1, -1) { $\adj{\holds{L}{5}{\alloc{\text{Bob}: 2}}, \;\holds{\text{Bob}}{4}{}, \; \holds{\text{Alice}}{3}{}}$ };

  \node (diag0) at (0, -2) {
    \begin{tikzpicture}[y=1cm, x=2cm]
      \node[sqadj] (a0) at (0, 0) {};
      \node[circB, label=below:Bob] (p0) at (0, -1) {};
      \draw[->] (a0) edge node[midway, left] { $4$ } (p0);

      \node[sqadj] (a0) at (1, 0) {};
      \node[circAB, label=right:L] (p0) at (1, -1) {};
      \node[circB, label=below:Bob] (p1) at (0.58, -2) {};
      \node[circA, label=below:Alice] (p2) at (1.42, -2) {};


      \draw[->] (a0) edge node[midway, left] { $5$ } (p0);
      \draw[->] (p0) edge node[midway, left] { $2$ } (p1);
      \draw[->] (p0) edge node[midway, left] { $3$ } (p2);

    \end{tikzpicture}
  };

  \node (diag1) at (1, -2) {
    \begin{tikzpicture}[y=1cm, x=2cm]
      \node[sqadj] (a0) at (0, 0) {};
      \node[circB, label=below:Bob] (p0) at (0, -1) {};
      \draw[->] (a0) edge node[midway, left] { $4$ } (p0);

      \node[sqadj] (a1) at (1, 0) {};
      \node[circAB, label=right:L] (pab) at (1, -1) {};
      \node[circB, label=below:Bob] (pabb) at (1, -2) {};
      \draw[->] (a1) edge node[midway, left] { $5$ } (pab);
      \draw[->] (pab) edge node[midway, left] { $2$ } (pabb);

      \node[sqadj] (a2) at (2, 0) {};
      \node[circA, label=below:Alice] (pa) at (2, -1) {};
      \draw[->] (a2) edge node[midway, left] { $3$ } (pa);

    \end{tikzpicture}
  };
\end{tikzpicture}
}
  \caption{
    A example of a redistribution operation from Turbo protocol.
    Here the transfer operation is used to move $3$ coins out of channel $L$ to Alice.
  }\label{fig:redistribution}
\end{figure}

The operations change the state of the adjudicator: typically both balances and outcomes.
There is no restriction on who can trigger the operations.
In this paper, we present all operations separately and assume they are called separately.
In practice, we expect that implementations would provide some utility methods that combine
common sequences of operations, to improve gas efficiency.

To recap, we now have a system where participants can deposit into state channel addresses on-chain.
By running a state channel, participants can reach an outcome and ensure this outcome is finalized on-chain.
Once the outcome is finalized, the funds held in the channel can be redistributed to other participants and channels by calling on-chain operations.
Participants can then withdraw any funds that have been redistributed to their address.

In the next section, we will use this model to develop some tools for constructing state channel networks and proving their safety.
