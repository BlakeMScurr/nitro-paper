\section{Reasoning about State Channels}\label{sec:reasoning}

- what does safety mean?

- paper aims to put in place a framework for reasoning about the correctness of state channel constructions
- reasoning about what I hold off-chain and what if means to me

- how to extract value
- states ---> outcomes ---> money
- reducing channel diagrams --> value

- outcome diagrams
- fundamental rule of state channels - if two states are worth the same, I will transition between them
- the "simple rule of state channels" - we ignore other factors
- value
- funded ?
- offloaded

- safe
- this is the guarantee that if a participant stops at any point other participants don't lose out

- allows rewriting
- [diagram] example: closing off-chain

- two questions (finalization + redistribution)
  - what can I definitely finalize?
  - what can I definitely redistribute to myself?

- protocol design: how can I move between states in single moves that keep the value the same
- .. when we don't allow atomic changes across channels

- presenting a construction / protocol
- in particular when presenting a protocol we must demonstrate a series single state updates, demonstrate that the value is preserved
- in particular the way we do this:
  - demostrate a construction funds a channel
  - demonstrate we can build it from another state
    - give a series of waypoint states - universally finalizable outcomes
    - of a special type of channel - consensus channels, running a particilar protocol
    - use the specifics of this channel to say we can move between them in a safe manner


\subsection{Finalizable outcomes}

- definition: statement
- definition: channel state
- definition: adjudicator state
- definition: system state

- the rules of strategies

subsection consensus game
[diagram] - pictorial representation of consensus game

\begin{figure}[h]\centering
  \makebox[\textwidth][c]{\begin{tikzpicture}[x=1.7cm, y=1cm]
  \begin{scope}[every node/.style={draw,circle,minimum size=4mm}]
    \node (A) at (0,0) {};
    \node (AB1) at (1,0) {};
    \node (AB2) at (2,0) {};
    \node[draw=none] (ABx) at (3,0) {};
    % Dots
    \foreach \n in {2.95,3,3.05} {\fill (\n,0) circle (0.8pt);}
    \node (ABn) at (4,0) {};
    \node (B) at (5,0) {};
  \end{scope}
  \begin{scope}
    \node at ([yshift=-0.5cm]A) { $A$ };
    \node at ([yshift=-0.5cm]AB1) { $(A, B, 1)$ };
    \node at ([yshift=-0.5cm]AB2) { $(A, B, 2)$ };
    \node at ([yshift=-0.5cm]ABn) { $(A, B, n-1)$ };
    \node at ([yshift=-0.5cm]B) { $B$ };
  \end{scope}
  \begin{scope}[-{Latex[length=2mm]}]
    \draw (A) to (AB1);
    \draw (AB1) to (AB2);
    \draw (AB2) to (ABx);
    \draw (ABx) to (ABn);
    \draw (ABn) to (B);
    \draw (AB1) to[bend right] (A);
    \draw (AB2) to[bend right] (A);
    \draw (ABn) to[bend right] (A);
  \end{scope}
\end{tikzpicture}
}
  \caption{Cool, huh?}
\end{figure}
